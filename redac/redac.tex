\documentclass[10pt,a4paper,french]{article}
\author{par Léo Peyronnet}
\title{Code Inverse, César et César affine}
\date{Décembre 2022}

\usepackage[utf8]{inputenc}
\usepackage[T1]{fontenc}

\usepackage{babel}
\usepackage{listings}
\usepackage{amsfonts}
\usepackage{amsmath}
\usepackage{amssymb}
\usepackage{amsthm}
\usepackage{mathtools}
\usepackage{tabto}
\usepackage{listings}
\usepackage{amssymb}
\usepackage{tabto}
\usepackage{xcolor}
\usepackage{hhline}

\definecolor{ao(english)}{rgb}{0.0, 0.5, 0.0}
\lstset{
  extendedchars=true,
  framexleftmargin=16pt,
  framextopmargin=3pt,
  framexbottommargin=6pt,
  frame=tb,
  commentstyle=\color{ao(english)},
  breaklines=true,
  language=Python,
  literate=
  {²}{{\textsuperscript{2}}}1
  {⁴}{{\textsuperscript{4}}}1
  {⁶}{{\textsuperscript{6}}}1
  {⁸}{{\textsuperscript{8}}}1
  {€}{{\euro{}}}1
  {é}{{\'e}}1
  {è}{{\`{e}}}1
  {ê}{{\^{e}}}1
  {ë}{{\¨{e}}}1
  {É}{{\'{E}}}1
  {Ê}{{\^{E}}}1
  {û}{{\^{u}}}1
  {ù}{{\`{u}}}1
  {â}{{\^{a}}}1
  {à}{{\`{a}}}1
  {á}{{\'{a}}}1
  {ã}{{\~{a}}}1
  {Á}{{\'{A}}}1
  {Â}{{\^{A}}}1
  {Ã}{{\~{A}}}1
  {ç}{{\c{c}}}1
  {Ç}{{\c{C}}}1
  {õ}{{\~{o}}}1
  {ó}{{\'{o}}}1
  {ô}{{\^{o}}}1
  {Õ}{{\~{O}}}1
  {Ó}{{\'{O}}}1
  {Ô}{{\^{O}}}1
  {î}{{\^{i}}}1
  {Î}{{\^{I}}}1
  {í}{{\'{i}}}1
  {Í}{{\~{Í}}}1,
}

\newenvironment{exemple}[1][]{\par\medskip
   \noindent \textbf{Exemple:} \rmfamily}{\medskip}

\begin{document}
\maketitle
\section{Rappel des méthodes}
\subsection{Code Inverse}\label{codeinv}
Cette méthode consiste à écrire le mot codé à l'envers. Pour ce faire, nous pouvons associer à chaque $x$ lettre du mot à crypter un entier $i \in [0,n[$ avec $n$ le nombre de lettres qui compose le mot. De fait, nous pouvons aussi associer à chaque $c$ lettre du mot crypté un entier $j \in [0,n[$.
Ainsi, la relation entre les deux mots est:
\begin{align*}
x_i \equiv c_{n-i-1} \Leftrightarrow c_j \equiv x_{n-j-1}
\end{align*}
Ainsi, le même algorithme peut être utilisé pour crypter et décrypter un mot.
\begin{exemple}
\begin{center}
\scriptsize
\begin{tabular}{|c|c|c|c|c|c|c|}
\hline
0 & 1 & 2 & $i$ & $n-3$ &$n-2$ & $n-1$ \\
\hline
C & R & Y & ... & A & G & E \\
\hhline{|=|=|=|=|=|=|=|}
$x_0=c_{n-1} $ & $x_1=c_{n-2} $ & $x_2=c_{n-3}$ & $x_i=c_{n-i-1}$ & $x_{n-3}=c_2$ & $x_{n-2}=c_1$ & $x_{n-1}=c_0$ \\
\hhline{|=|=|=|=|=|=|=|}
E & G & A & ... & Y & R & C \\
\hline
0 & 1 & 2 & $j$ & $n-3$ &$n-2$ & $n-1$ \\
\hline
\end{tabular}
\normalsize
\end{center}
\end{exemple}
\subsection{Code César}\label{cesar}
Cette méthode consiste à décaler chaque lettre d'un mot par rapport à un alphabet. Pour ce faire, pour un mot $x$, déterminons $\alpha$ un alphabet incluant toutes les lettres de $x$ et $n$ la taille de $\alpha$. Soit $i \in [0,p[$ l'index d'une lettre de $x$ et $p$ la taille du mot, alors $x_i$ correspond à l'indice de cette lettre dans $\alpha$. Ainsi, pour $b \in \mathbb{Z}$ le décalage à appliquer dans le code, nous avons $y$ le mot codé tel que:
\begin{align*}
y_i \equiv x_i+b [n]
\end{align*}
Le décryptage avec $x$ le mot crypté et $y$ le mot décrypté correspond alors à:
\begin{align*}
y_i \equiv x_i-b [n]
\end{align*}

\subsection{Code César affine}\label{cesaraff}
Cette méthode consiste à appliquer à chaque lettre d'un mot une autre valeur via une fonction affine. Soit $x$,$y$,$i$,$\alpha$ et $n$ comme vu à la méthode précédente. Soit $a$ et $b$ des entiers relatifs, alors:
\begin{align*}
y_i \equiv ax_i+b [n]
\end{align*}
Le décryptage avec $x$ le mot crypté et $y$ le mot décrypté correspond alors à:
\begin{align*}
y_i \equiv a'(x_i-b) [n]
\end{align*}
avec $a'$ l'inverse modulaire de $a$.

\section{Présentation des programmes}
\subsection{codeInverse()}
Fonction relative à la méthode Code Inverse (cf. \ref{codeinv}).
\begin{lstlisting}
def codeInverse(mot):
    result=""
    for i in range(len(mot)-1,-1,-1): # len(mot)-1 >= i >= 0
        # soit j un compteur incrémenté à chaque passage de boucle (sousentendu ici car boucle for)
        result+=mot[i] # la i-ème lettre du mot en clair devient la j-ème lettre du mot crypté
    return result
\end{lstlisting}
\subsection{scan()}
Fonction retournant l'indice d'une lettre dans un alphabet. Utilisé pour la méthode César et César affine.
\begin{lstlisting}
def scan(lettre, alphabet):
    y=0
    while (lettre!=alphabet[y]): #si la i-ème lettre du mot est dans l'alphabet (à l'index y), alors fin de boucle
            y+=1 #passage à la lettre suivante de l'alphabet
            if y==len(alphabet): #si la i-ème lettre n'est pas dans l'alphabet
                print("attention: un des caractères du mot n'est pas dans l'alphabet !")
                exit()
    return y
\end{lstlisting}
\subsection{codeCesar() et decodeCesar()}
Fonction de codage et décodage relative à la méthode Code César (cf. \ref{cesar}).
\begin{lstlisting}
def codeCesar(alphabet,mot,cle):
    result=""
    for i in range(len(mot)): # 0 <= i <= len(mot)-1
        y=scan(mot[i],alphabet) # position de la i-ème lettre du mot dans l'alphabet ; copions la valeur de y à cette étape sous le nom x
        y=(y+cle)%len(alphabet) # décalage avec la clé cézar (y congru à x + cle modulo la taille de l'alphabet)
        result+=alphabet[y] #la i-ème lettre du mot en clair d'index x dans l'alphabet devient la i-ème lettre du mot crypté d'index y. 
    return result
    
def decodeCesar(alphabet,mot,cle):
    result=""
    for i in range(len(mot)): # 0 <= i <= len(mot)-1
        y=scan(mot[i],alphabet) # position de la i-ème lettre du mot dans l'alphabet ; copions la valeur de y à cette étape sous le nom x
        y=(y-cle)%len(alphabet) # décalage avec la clé cézar (y congru à x - cle modulo la taille de l'alphabet)
        result+=alphabet[y] #la i-ème lettre du mot en clair d'index x dans l'alphabet devient la i-ème lettre du mot crypté d'index y. 
    return result
\end{lstlisting}
\subsection{codeCesarAff()}
Fonction de codage relative à la méthode Code César affine (cf. \ref{cesaraff}).
\begin{lstlisting}
def codeCesarAff(alphabet,mot,a,b):
    result=""
    for i in range(len(mot)): # 0 <= i <= len(mot)-1
        y=scan(mot[i],alphabet) # position de la i-ème lettre du mot dans l'alphabet ; copions la valeur de y à cette étape sous le nom x
        y=(a*y+b)%len(alphabet) # cryptage affine (y congru à a*x+b modulo la taille de l'alphabet)
        result+=alphabet[y] #la i-ème lettre du mot en clair d'index x dans l'alphabet devient la i-ème lettre du mot crypté d'index y. 
    return result
\end{lstlisting}
\subsection{euclideEtt() et inv()}
Fonctions employées par decodeCesarAff() (cf. \ref{cesaraffc}):
\begin{itemize}
\item euclideEtt(): Algorithme d'Euclide étendu. Renvoie l'identité de Bézout ($a.u+b.v=pgcd(a,b)$) sous forme d'une liste de trois entiers. r1: pgcd; u1: u et v1: v.
\item inv(): Revoie u si r1=1.
\end{itemize}
\begin{lstlisting}
def euclideEtt(a,b):
    r1=b
    r2=a
    u1=0
    v1=1
    u2=1
    v2=0
    while r2!=0:
        q=r1//r2
        
        r3=r1
        u3=u1
        v3=v1

        r1=r2
        u1=u2
        v1=v2

        r2=r3-q*r2
        u2=u3-q*u2
        v2=v3-q*v2
    return [r1,u1,v1] # retourne l'identité de bézout (r1:pgcd;u1:u et v1:v)

def inv(a,modulo):
    t=euclideEtt(a,modulo) #import de l'identité de bézout
    if t[0]==1: #si le pgcd=1
        return t[1] # retourne l'inverse modulaire de a :(u)
    else: #si le pgcd!=1
        print("attention: le coefficient A de la fonction affine n'est pas premier avec la taille de l'alphabet !")
        exit()
\end{lstlisting}
\newpage
\subsection{decodeCesarAff()}\label{cesaraffc}
Fonction de décodage relative à la méthode Code César affine (cf. \ref{cesaraff}).
\begin{lstlisting}
def decodeCesarAff(alphabet,mot,a,b):
    result=""
    for i in range(len(mot)): # 0 <= i <= len(mot)-1
        y=scan(mot[i],alphabet)  # position de la i-ème lettre du mot dans l'alphabet ; copions la valeur de y à cette étape sous le nom x
        y=(inv(a,len(alphabet))*(y-b))%len(alphabet) # décryptage affine (y congru à a'*(x-b) modulo la taille de l'alphabet)
        result+=alphabet[y] #la i-ème lettre du mot en clair d'index x dans l'alphabet devient la i-ème lettre du mot crypté d'index y.
    return result
\end{lstlisting}
\end{document}